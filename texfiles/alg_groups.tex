\documentclass{memoir}

% Imports

%% Quotations (S. Gammelgaard)
\usepackage{verbatim}
\usepackage{csquotes}

%% Mathematics
\usepackage{amsfonts}
\usepackage{amsmath}
\usepackage{amssymb}    % Extra symbols
\usepackage{amsthm}     % Theorem-like environments
\usepackage{calligra}	% For the \sheafHom command
\usepackage{cancel}     % Cancel terms with \cancel, \bcancel or \xcancel
\usepackage{dsfont}     % Double stroke font with \mathds{}
\usepackage{mathtools}  % Fonts and environments for mathematical formulae
\usepackage{mathrsfs}   % Script font with \mathscr{}
\usepackage{stmaryrd}   % Brackets
\usepackage{thmtools}   % Theorem-like environments, extends amsthm

%% Graphics
\usepackage[dvipsnames,svgnames,cmyk]{xcolor}     % Pre-defined colors
\usepackage{graphicx}         % Tool for importing images
\graphicspath{{figures/}}
\usepackage{tikz}             % Drawing tool
\usetikzlibrary{calc}
\usetikzlibrary{intersections}
\usetikzlibrary{decorations.markings}
\usetikzlibrary{arrows}
\usetikzlibrary{positioning}
\usepackage{tikz-cd}		  % Commutative diagrams
\usepackage[all]{xy}

%% Organising tools
\usepackage[notref, notcite]{showkeys}               % Labels in margins
\usepackage[color= LightGray,bordercolor = LightGray,textsize    = footnotesize,figwidth    = 0.99\linewidth,obeyFinal]{todonotes} % Marginal notes

%% Misc
\usepackage{xspace}         % Clever space
\usepackage{textcomp}       % Extra symbols
\usepackage{multirow}       % Rows spanning multiple lines in tables
\usepackage{tablefootnote}  % Footnotes for tables

%% Bibliography
\usepackage[backend = biber, style = alphabetic, ibidtracker=true]{biblatex}
\addbibresource{bibliography.bib}

%% Cross references
\usepackage{varioref}

\usepackage[pdftex,hidelinks]{hyperref}
\usepackage[nameinlink, capitalize, noabbrev]{cleveref}

\pageaiv
\stockaiv

\setsecnumdepth{subsection}

\pretitle{\begin{center}\huge\sffamily\bfseries}

%% Book
\renewcommand*{\printbooktitle}[1]
{
    \hrule\vskip\onelineskip
    \centering\booktitlefont #1
    \vskip\onelineskip\hrule
}
\renewcommand*{\afterbookskip}{\par}
\renewcommand*{\booktitlefont}{\Huge\bfseries\sffamily}
\renewcommand*{\booknamefont}{\normalfont\huge\bfseries\MakeUppercase}


%% Part
\renewcommand*{\printparttitle}[1]
{
    \hrule\vskip\onelineskip
    \centering\parttitlefont #1
    \vskip\onelineskip\hrule
}
\renewcommand*{\afterpartskip}{\par}
\renewcommand*{\parttitlefont}{\Huge\bfseries\sffamily}
\renewcommand*{\partnamefont}{\normalfont\huge\bfseries\MakeUppercase}


%% Chapter 
\makeatletter
\chapterstyle{demo2}
\renewcommand*{\printchaptername}
{
    \centering\chapnamefont\MakeUppercase{\@chapapp}
}
\renewcommand*{\printchapternum}{\chapnumfont\thechapter\space}
\renewcommand*{\chaptitlefont}{\Huge\bfseries\sffamily\center}
\let\ps@chapter\ps@empty


%% Lower level sections
\setsecheadstyle{\Large\bfseries\sffamily\raggedright}
\setsubsecheadstyle{\large\bfseries\sffamily\raggedright}
\setsubsubsecheadstyle{\normalsize\bfseries\sffamily\raggedright}
\setparaheadstyle{\normalsize\bfseries\sffamily\raggedright}
\setsubparaheadstyle{\normalsize\bfseries\sffamily\raggedright}


%% Abstract
\renewcommand{\abstractnamefont}{\sffamily\bfseries}


%% Header
\pagestyle{ruled}
\makeevenhead{ruled}{\sffamily\leftmark}{}{}
\makeoddhead {ruled}{}{}{\sffamily\rightmark}


%% Trim marks
\trimLmarks

%% Environments
\declaretheorem[style = plain, numberwithin = section]{theorem}
\declaretheorem[style = plain,      sibling = theorem]{corollary}
\declaretheorem[style = plain,      sibling = theorem]{lemma}
\declaretheorem[style = plain,      sibling = theorem]{proposition}
\declaretheorem[style = plain,      sibling = theorem]{observation}
\declaretheorem[style = plain,      sibling = theorem]{conjecture}
\declaretheorem[style = definition, sibling = theorem]{definition}
\declaretheorem[style = definition, sibling = theorem]{example}
\declaretheorem[style = definition, sibling = theorem]{notation}
\declaretheorem[style = remark,     sibling = theorem]{remark}
\crefname{observation}{Observation}{Observations}
\Crefname{observation}{Observation}{Observations}
\crefname{conjecture}{Conjecture}{Conjectures}
\Crefname{conjecture}{Conjecture}{Conjectures}
\crefname{notation}{Notation}{Notations}
\Crefname{notation}{Notation}{Notations}
\crefname{diagram}{Diagram}{Diagrams}
\Crefname{diagram}{Diagram}{Diagrams}

%% Operators
\DeclareMathOperator{\Spn}{Span}				% Span of vectors
\DeclareMathOperator{\Gal}{Gal}					% Galois group
\DeclareMathOperator{\Spec}{Spec}				% Spectrum
\DeclareMathOperator{\Proj}{Proj}				% Proj construction
\DeclareMathOperator{\Gr}{\mathbb{G}}			% Grassmannian
\DeclareMathOperator{\Aut}{Aut}					% Automorphisms
\DeclareMathOperator{\End}{End}					% Endomorphisms
\DeclareMathOperator{\CH}{CH}					% Chow ring/group
\DeclareMathOperator{\CHr}{CH^\bullet}			% Chow ring
\DeclareMathOperator{\Cox}{Cox}					% Cox ring
\DeclareMathOperator{\Div}{Div}					% Divisor group
\DeclareMathOperator{\Cl}{Cl}					% Class group
\DeclareMathOperator{\Pic}{Pic}					% Picard group
\DeclareMathOperator{\relSpec}{\mathbf{Spec}}	% Relative Spec
\DeclareMathOperator{\relProj}{\mathbf{Proj}}	% Relative Proj
\DeclareMathOperator{\ord}{ord}					% Order
\DeclareMathOperator{\res}{res}					% Residue
\DeclareMathOperator{\coker}{coker}				% Cokernel (\ker is already defined)
\DeclareMathOperator{\im}{im}					% Image
\DeclareMathOperator{\coim}{coim}			    % Coimage
\DeclareMathOperator{\tr}{tr}					% Trace
\DeclareMathOperator{\rk}{rk}					% Rank
\DeclareMathOperator{\Hom}{Hom}					% Homomorphisms
\DeclareMathOperator{\cl}{cl}					% Class map
\DeclareMathOperator{\sheafHom}					% Sheaf of homomorphisms
{
    \mathscr{H}\text{\kern -5.2pt {\calligra\large om}}\,
}
\DeclareMathOperator{\codim}{codim}				% Codimension
\DeclareMathOperator{\Sym}{Sym}					% Symmetric powers
\DeclareMathOperator{\II}{I\!I}					% Second fundamental form
\DeclareMathOperator{\Pfaff}{Pfaff}				% Pfaffian

%% Delimiters
\DeclarePairedDelimiter{\p}{\lparen}{\rparen}          % Parenthesis
\DeclarePairedDelimiter{\set}{\lbrace}{\rbrace}        % Set
\DeclarePairedDelimiter{\abs}{\lvert}{\rvert}          % Absolute value
\DeclarePairedDelimiter{\norm}{\lVert}{\rVert}         % Norm
\DeclarePairedDelimiter{\ip}{\langle}{\rangle}         % Inner product, ideal
\DeclarePairedDelimiter{\sqb}{\lbrack}{\rbrack}        % Square brackets
\DeclarePairedDelimiter{\ssqb}{\llbracket}{\rrbracket} % Double brackets
\DeclarePairedDelimiter{\ceil}{\lceil}{\rceil}         % Ceiling
\DeclarePairedDelimiter{\floor}{\lfloor}{\rfloor}      % Floor
\DeclarePairedDelimiter{\tuple}{\langle}{\rangle}		% Tuple	


%% Sets
\newcommand{\N}{\mathbb{N}}    						% Natural numbers
\newcommand{\Z}{\mathbb{Z}}    						% Integers
\newcommand{\Q}{\mathbb{Q}}    						% Rational numbers
\newcommand{\R}{\mathbb{R}}    						% Real numbers
\newcommand{\C}{\mathbb{C}}    						% Complex numbers
\newcommand{\A}{\mathbb{A}}    						% Affine space
\renewcommand{\P}{\mathbb{P}}  						% Projective space
%Additions (S. Gammelgaard)
\renewcommand{\H}{\mathbb{H}}						% Hyperbolic space, or half-plane
\newcommand{\D}{\mathbb{D}} 						% Unit disk
\newcommand{\F}{\mathbb{F}} 						% Field
\newcommand{\bP}[1]{\mathbf{P}\!\left(#1\right)}	% Projectivisation of bundles

%% Special groups and Lie groups
\newcommand{\GL}{\mathbf{GL}}						% General linear group
\newcommand{\PGL}{\mathbf{PGL}}						% Projective linear group
\newcommand{\PSL}{\mathbf{PSL}}						% Projective linear group
\newcommand{\SL}{\mathbf{SL}}						% Special linear group
\newcommand{\SO}{\mathbf{SO}}						% Special linear group
\newcommand{\Mat}{\mathbf{Mat}}						% Special linear group

%% Lie algebras
\newcommand{\lalg}[1]{{\normalfont\mathfrak{#1}}}	% General for Lie algebras
\newcommand{\gl}{\lalg{gl}}							% General linear algebra
%\newcommand{\sl}{\lalg{sl}}							% Special linear algebra

%% Cones of cycles on varieties and related objects
\newcommand{\NS}{\mathrm{NS}}						% Neron-Severi group
\newcommand{\Nef}{\mathrm{Nef}}						% Nef cone
\newcommand{\NE}{\mathrm{NE}}						% Cone of curves
\newcommand{\Eff}{\mathrm{Eff}}						% Effective cone
\newcommand{\Pseff}{\mathrm{PSeff}}					% Pseudoeffective cone

%% Categories
\newcommand{\cat}[1]{{\normalfont\mathsf{#1}}}	% General for categories
\newcommand{\Cat}{\cat{Cat}}					% Category of categories
\newcommand{\Sch}{\cat{Sch}}					% Schemes
\newcommand{\Set}{\cat{Set}}					% Sets
\newcommand{\Grp}{\cat{Grp}}					% Groups
\newcommand{\AbGrp}{\cat{AbGrp}}				% AbGroups
\newcommand{\Ab}{\cat{Ab}}      				% AbGroups
\newcommand{\Ring}{\cat{Ring}}					% Rings
\newcommand{\Top}{\cat{Top}}					% Topological spaces
\newcommand{\Alg}{\cat{Alg}}					% Algebras
\newcommand{\SMan}{\cat{Man}^\infty}			% Smooth manifolds
\newcommand{\Coh}[1]{\cat{Coh}({#1})}			% Coherent sheaves
\newcommand{\QCoh}[1]{\cat{QCoh}({#1})}			% Quasi-coherent sheaves
\newcommand{\Fun}{\cat{Fun}}					% Category of functors
\newcommand{\PreSh}{\cat{PreSh}}			    % Category of presheaves
\newcommand{\Sh}{\cat{Sh}}			            % Category of presheaves

%% Miscellaneous mathematics
\newcommand{\ol}[1]{\overline{#1}}							% Overline
\newcommand{\Dirsum}{\bigoplus}								% Direct sum
\newcommand{\shf}[1]{\mathscr{#1}}							% Sheaf
\newcommand{\OO}{\mathcal{O}}								% Structure sheaf
\DeclareMathOperator{\id}{id}								% Identity
\newcommand{\tens}[1]{\otimes_{#1}}							% Tensor product
\newcommand{\normal}{\vartriangleleft}						% Normal subgroup, ideal of ring or Lie algebra
\newcommand{\lamron}{\vartriangleright}						% The opposite of above
\newcommand{\dvol}{d\operatorname{vol}}						% Volume form on a KÀhler manifold
\newcommand{\cha}{\operatorname{char}}						% Characteristic of a field
\newcommand{\Hilb}{\operatorname{Hilb}}						% Hilbert scheme
\newcommand{\isoto}{\xrightarrow{\sim}}						% Isomorphism
\newcommand{\injto}{\xhookrightarrow{}}						% Injective map
\newcommand{\ratto}{\dashrightarrow}						% Rational map
\newcommand{\rateq}{\overset{\sim}{\ratto}}					% Rational equivalence
\newcommand{\Bl}[2]{\operatorname{Bl}_{#2} #1}				% Blow-up of #1 along #2
%\newcommand{\Bl}[2]{#1\kern -2pt \uparrow_{#2}}			% 	(alternativ som ingen andre liker, buhu)
\newcommand{\fracpart}[2]{\frac{\partial #1}{\partial #2}}	% Partial derivative
\renewcommand{\setminus}{\smallsetminus}
\newcommand{\transp}[1]{{}^t#1}								% transposed map, Voisin-style
\newcommand{\dual}{{}^\vee}									% dual of map, vector bundle, sheaf, etc...
\newcommand{\littletilde}{\tilde}							% for the next
\renewcommand{\tilde}{\widetilde}
\newcommand{\Spe}{\text{Sp\'e}}						        % Etale space
\newcommand{\Gm}{{\mathbb{G}_m}}							% Multiplicative group
\newcommand{\Ga}{\mathbb{G}_a}								% Additive group
\newcommand{\actson}{\curvearrowright}						% Group acting on

%%\newcommand{\dual}{{}^{\smash{\scalebox{.7}[1.4]{\rotatebox{90}{\guilsinglleft}}}}}	% Dual of sheaf/vector space et cetera

%% Miscellaneous, not-strictly-mathematical
\renewcommand{\qedsymbol}{\(\blacksquare\)}
\newcommand{\ie}{\leavevmode\unskip, i.e.,\xspace}
\newcommand{\eg}{\leavevmode\unskip, e.g.,\xspace}
%\newcommand{\wlog}{\leavevmode\unskip without loss of generality \xspace}
\newcommand{\dash}{\textthreequartersemdash\xspace}
\newcommand{\TikZ}{Ti\textit{k}Z\xspace}
\newcommand{\matlab}{\textsc{Matlab}\xspace}


\title{Algebraic groups}

\author{Emile T. Okada,\\ S\o ren Gammelgaard,\\ Esteban Gomezllata Marmolejo}

\begin{document}
\maketitle
\tableofcontents
\section{Conventions}
Categories are usually denoted by \emph{sans-serifs} \ie $ \Grp, \Top, \Sch $ are the categories of groups, topological spaces, and schemes, respectively. 

After a while, given a Hopf algebra $ A $, we may use $ \Spec A $ for the affine group scheme represented by $ A $, and we may similarly write "the coordinate ring" to mean the Hopf algebra representing some affine group scheme. 

The notation $ (-)\dual $ means \emph{dual}, where $ (-) $ stands for a group scheme, vector space, line bundle \&c.
\chapter{Affine group schemes}
\section{Affine schemes}
\begin{definition}
    Let $k$ be a commutative ring. An \textit{affine scheme over $k$} is a representable functor $F:\Alg_k\to \Set$.
\end{definition}
Let us try to motivate this definition a bit. 
Let $k$ be a commutative ring and $\{f_\alpha\}\subseteq k[x_1,\dots,x_n]$.
For any $k$-algebra $R$ we can consider the set of points in $R^n$ satisfying $f_\alpha = 0$ for all $\alpha$.
Let us call this set $I(R)$.
It is clear that if we have a $k$-algebra homomorphism $f:R\to S$ we obtain a map $I(f):I(R) \to I(S)$, and that this turns $I$ into a functor $I:\Alg_k\to \Set$.
\begin{theorem}
    Let $J$ be the ideal of $k[x_1,\dots,x_n]$ generated by the $f_\alpha$ and $A = k[x_1,\dots,x_n]/J$.
    Then $I$ is a representable functor with representative $A$.
\end{theorem}
\begin{proof}
    Let $a = (a_1,\dots,a_n)\in I(R)$ and let $f_a:k[x_1,\dots,x_n]\to R$ be the $k$-algebra homomorphism given by $x_i\mapsto a_i$.
    By definition of the point $a$, the ideal $J$ must map to 0 under $f_a$.
    Thus $f_a$ factors through $A$ and we obtain a map $\bar f_a: A\to R$.

    Conversely, given a map $f:A\to R$, let $a = (f(\bar x_1),\dots,f(\bar x_n))\in R^n$ where $\bar x_i$ is the image of $x_i$ in $A$.
    Since $f$ is a homomorphism it is clear that $a$ lies in $I(R)$.

    These two maps give a bijection between the sets $I(R)$ and $\Hom_k(A,R)$.
    It is straightforward to check this is a natural bijection and so $I$ is naturally isomorphic to the functor $\Hom(A,-)$.
\end{proof}
This theorem says that the affine schemes are exactly what we expect them to be.
\subsection{The Yoneda lemma}
Recall from category theory the following result on representable functors.
\begin{theorem}
    (Yoneda's lemma).
    Let $F:\mathcal C\to \Set$ be a functor, $\mathcal C$ be locally small and $c\in \mathcal C$. Then there is a natrual bijection
    \begin{align}
        \Hom(\mathcal C(c,-),F) &\leftrightarrow Fc \\
        \eta \quad &\rightarrow \quad \eta_c(\id_c) \nonumber \\
        \eta_x: f\mapsto F(f)(y) \quad &\leftarrow \quad y. \nonumber 
    \end{align}
\end{theorem}
\begin{corollary}
    (Yoneda embedding).
    The functor 
    \begin{equation}
        \mathcal C(\bullet,-):\mathcal C^{op}\to \Fun(\mathcal C, \Set)
    \end{equation}
    is full and faithful.
\end{corollary}
\begin{proof}
    Let $c,d\in \mathcal C$.
    From the Yoneda lemma we have a bijection
    \begin{equation}
        \Hom(\mathcal C(d,-),\mathcal C(c,-)) \leftrightarrow \mathcal C(c,d).
    \end{equation}
    Let $f\in \mathcal C(c,d)$. 
    Under the bijection this gets sent to the natural transformation $\eta_x: g\mapsto \mathcal C(c,-)(g)(f) = g\circ f$.
    Thus the backwards map in the Yoneda bijection is just $\mathcal C(\bullet, -)$ and so the result follows.
\end{proof}
\begin{remark}
    It follows that $\mathcal C^{op}$ is equivalent to the category of representable functors from $\mathcal C$.
    In fact there exists a functor $P:\Fun^{rep}(\mathcal C,\Set) \to \mathcal C^{op}$ such that $P\circ \mathcal C(\bullet,-) = \id_{\mathcal C^{op}}$, $\mathcal C(\bullet,-)\circ P \cong \id_{\Fun^{rep}(\mathcal C,\Set)}$ and $\mathcal C(\bullet,-)\circ P|_{\im(\mathcal C(\bullet,-)} = \id_{\im(\mathcal C(\bullet,-)}$.
\end{remark}
\begin{corollary}
    Let $c,d\in \mathcal C$. 
    Then $c\cong d$ if and only if $\mathcal C(c,-) \cong \mathcal C(d,-)$.
\end{corollary}
\begin{proof}
    $(\Rightarrow)$ This follows by functoriality.

    $(\Leftarrow)$ Let $\alpha: \mathcal C(c,-) \Rightarrow \mathcal C(d,-)$ be an isomorphism and $\beta$ its inverse.
    Let $a:d\to c$ and $b:c\to d$ be the corresponding maps.
    By naturality of the Yoneda bijection
    \begin{equation}
        \begin{tikzcd}
            \Hom(\mathcal C(c,-),\mathcal C(d,-)) \arrow[r] \arrow[d,"\beta\circ"] & \mathcal C(d,c) \arrow[d,"\beta_c"] \\
            \Hom(\mathcal C(c,-),\mathcal C(c,-)) \arrow[r] & \mathcal C(c,c) 
        \end{tikzcd}
    \end{equation}
    commutes.
    So $\beta_c(\alpha_c(\id_c)) = \id_c$.
    But $\beta_c(\alpha_c(\id_c)) = a\circ b$ and so $a\circ b = \id_c$.
    Similarly $b\circ a = \id_b$ and so the result follows.

    Alternatively note use the remark. 
\end{proof}
This last corollary implies that two $k$-algebras are isomorphic if and only if the corresponding affine schemes are.
\section{Affine group schemes}
\begin{definition}
    An \textit{affine group scheme over $k$} is a functor $F:\Alg_k\to \Grp$ such that $F$ composed with the forgetful functor $\Grp \to \Set$ is representable.
\end{definition}
\begin{example}
    Let $k = \Z$ (so $\Alg_k = \Ring$).
    Then $\SL_n:\Ring\to \Grp$, $R\mapsto \SL_n(R)$ is an affine group scheme over $\Z$.
\end{example}
\begin{proposition}
    Let $E,F,G$ be affine group schemes over $k$ represented by $A,B,C$ respectively.
    If we have morphisms $E\to G$ and $F\to G$ then the pullback exists and is represented by $A\otimes_CB$.
\end{proposition}
\begin{proof}
    Existence follows from the fact that we can compute the pullback pointwise and $\Grp$ has pullbacks.
    Explicitly 
    \begin{equation}
        (E\times_G F)(R) = \{(e,f) : \text{ $e$ and $f$ have same image in $G(R)$}\}.
    \end{equation}
    For the second part of the proposition note that the pullback in $\Alg_k^{op}$ is of the required form and that we have an equivalence of categories between $\Alg_k^{op}$ and affine schemes over $k$.
\end{proof}
\begin{definition}
    Let $F$ be an affine group scheme over $k$ and let $\phi:k\to k'$ be a ring homomorphism.
    Any $k'$-algebra can be turned into a $k$ by composing by $\phi$ and so we can turn $F$ into a functor on $k'$-algebras.
    Call this new functor $F_{k'}$.
\end{definition}
\begin{proposition}
    If $F$ is represented by $A$ then $F_{k'}$ then is represented by $A\otimes_kk'$.
\end{proposition}
\begin{proof}
    There is a natural bijection 
    \begin{equation}
        \Hom_{k'}(A\otimes_kk',S) \leftrightarrow \Hom_k(A,S).
    \end{equation}
\end{proof}
\subsection{Hopf algebras}
Another way to define an affine group scheme is that it is a representable functor $F:\Alg_k \to \Set$ together with natural transformations $\mu:F\times F\to F$, $i:F\to F$ and $u:e\to G$ where $e$ is the functor $\Hom_k(k,-)$ such that
\begin{equation}
    \begin{tikzcd}
        F\times F\times F \arrow[r,"\id\times\mu"] \arrow[d,"\mu\times\id"] & F\times F \arrow[d,"\mu"] \\
        F\times F \arrow[r,"\mu"] & F
    \end{tikzcd}
\end{equation}
\begin{equation}
    \begin{tikzcd}
        e\times F \arrow[r,"u\times\id"] \arrow[dr,"\cong"] & F\times F \arrow[d,"\mu"] & F\times e \arrow[r,"\id\times u"] \arrow[dr,"\cong"] & F\times F \arrow[d,"\mu"] \\
                                                            & F & & F
    \end{tikzcd}
\end{equation}
\begin{equation}
    \begin{tikzcd}
        F \arrow[r,shift right,swap,"i\times\id"] \arrow[r,shift left,"\id\times i"] \arrow[d] & F\times F \arrow[d,"\mu"] \\
        e \arrow[r,"u"] & F
    \end{tikzcd}
\end{equation}
all commute.
If $F$ is represented by $A$ then translating back to $\Alg_k$ we obtain maps $\Delta:A\to A\otimes_k A$, $S:A\to A$ and $\epsilon:A\to k$.
Satisfying the same commutative diagrams, but with arrows reversed and $\mu$ replaced with $\Delta$ etc.
The algebra $A$ together with these maps is what we call a Hopf algebra.
\begin{definition}
    Let $\psi:H\to G$ be a morphism of affine groups schemes. 
    We say $\psi$ is a \textit{closed embedding} if the corresponding map on algebras is surjective.
    $H$ is then isomorphic to a closed subgroup of $G$ represented by the corresponding quotient of the algebra of $A$.
\end{definition}
\begin{theorem}
    Affine group schemes over $k$ correspond to Hopf algebras over $k$.
\end{theorem}
\begin{definition}
    Let $A$ be a $k$-algebra. 
    We call an ideal $I$ of $A$ a \textit{Hopf ideal} if $A/I$ inherits the structure of a Hopf algebra.
\end{definition}
\begin{proposition}
    Let $A$ be a $k$-algebra and $I\normal A$. Then $I$ is a hopf ideal if and only if $\Delta(I)\subseteq I\otimes A + A\otimes I$, $S(I)\subseteq I$ and $\epsilon(I) = 0$.
\end{proposition}
\begin{proof}
    Todo.
\end{proof}
\begin{definition}
    Let $\Phi:G\to H$ be a morphism of affine groups schemes. 
    Then $\ker\Phi (R) = \ker(G(R)\to H(R))$ or alternatively $\ker\Phi = G\times_H\{e\}$.
    It follows that if $G,H$ are represented by $A,B$ respectively then $\ker\Phi$ is represented by $A\otimes_B k\cong A/AI_B$.
\end{definition}
\subsection{Characters}
\begin{definition}
    A homomorphism $G\to G_m$ is called a character.
\end{definition}
\begin{theorem}\label{thm:characters_are_gp-like_elems}
    The characters of an affine group scheme $G$ represented by $A$ correspond to the group-like elements of $A$.
\end{theorem}
\section{Diagonalisable group schemes}\label{sec:diag_gp_sch}

Let $ M $ be an abelian group, $ k $ a ring, and $ k[M] $ the group algebra. We construct a Hopf algebra structure on $ k[M] $ by making the group elements group-like \ie we set \[ \Delta(m)=m\otimes m,\quad \epsilon(m)=1,\quad S(m)=m^{-1}, \] and we extend by linearity to all $ k[M] $. This is a Hopf algebra (easily checked on elements)
\begin{definition}\label{def:diag_gp_sch}
	A group scheme constructed in this way is called a \emph{diagonalisable group scheme}.
\end{definition}
The name is not immediately a good one, but we will later see that most affine group schemes can be embedded into $ \mathbf{GL}_n $ for some $ n $ (\cref{thm:everything_in_GL}), and that the diagonalisable group schemes will be precisely those corresponding to groups of diagonalisable matrices. 

We can describe these schemes completely:
\begin{theorem}\label{thm:struct_of_diag_gps}
	Let $ G $ be a diagonalisable group scheme, represented by $ A $, and suppose that $ A $ is a finitely generated $ k $-algebra.
	Then
	\[ G=\left(\prod_{i=1}^{n}\Gm\right)\times\prod_{j=1}^{n'}\pmb{\mu}_{m_j} \] for some integers $ n,n', m_j$. 
\end{theorem}
\begin{proof}
	Take a finite set of generators of $ A $. These are finite linear combinations of elements of $ M $, so we have a finite number of elements of $ M $ generating $ A $, Then these elements must also generate $ M $. So $ M $ is a finitely generated abelian group. But such groups are direct sums of copies of $ \Z $ and $ \Z/n\Z $, so we need only check two cases.
	
	Case 1: $ M=\Z $. Then $ k[\Z] $ has basis $ \{e_n|n\in \Z\} $, with multiplication $ e_i e_j = e_{i+j} $. So $ k[\Z]\isoto k[X, X^{-1}] $, the affine coordinate ring of $ \Gm $.

	Case 2: $ M=\Z/n\Z $. The basis is now $ \{1=e_0, e_1,\dots, e_{n-1}=e^{n-1}\} $. So $ e_1^n=1 $, and the Hopf algebra is $ k[X]/(X^n-1) $, which represents $ \pmb\mu_n $.
\end{proof}
We will now really exploit the Hopf algebra structure. First, we have
\begin{lemma}\label{lem:gplikes_lin_indep}
	The group-like elements in a Hopf algebra over a field $ k $ are linearly independent.
\end{lemma}
\begin{proof}
	Take group-like elements $ b, \{b_i\} $ such that \[ b=\sum\lambda_ib_i\qquad\textrm{(a finite sum),} \]
	and assume the $ b_i $ to be independent. Then
	\[ 1=\epsilon(b)=\sum\lambda_i\epsilon(b_i)=\sum\lambda_i. \]
	But \begin{align*} \Delta(b)&=b\otimes b=\sum \lambda_i\lambda_jb_i\otimes b_j\\
	&=\sum\lambda_i\Delta(b_i)=\sum\lambda_ib_i\otimes b_i\end{align*} so all $ \lambda_i\lambda_j=0 $ for $ j\ne i $, $ \lambda_i^2=\lambda_i $, so all $ \lambda_i $ are either 0 or 1. Since their sum is 1, exactly one $ \lambda_i =1$, and so $ b=b_i $.
\end{proof}
Furthermore:
\begin{theorem}\label{thm:diagon_implies_cring_is_span_of_gplikes}
	An affine group scheme over a field is diagonalisable iff its representing algebra (coordinate ring!) is spanned by group-like elements. There is a natural coequivalence of categories\[ \{\textrm{Diagonalisable group schemes}/k\}\isoto \cat{AbGrp}. \]
\end{theorem}
\begin{proof}
	By the lemma, if $ A $ is spanned by group-like elements, they are a basis of $ A $. The group formed by them is the character group $ X_G $. So $ k[X_G]\isoto A $, at least as sets. Checking on basis elements shows that this is an isomorphism of Hopf algebras.
	If $ M $ is any abelian group, its elements are the only group-like elements of $ k[M] $. Thus $ M $ is the character group of $ \Spec k[M] $ (the group scheme rep. by $ k[M] $).
	Also, if $ G\to H $ is a homomorphism of affine algebraic groups, we have an induced map $ X_H\to X_G $. This determines the Hopf algebra structure, since $ X_H $ spans $ k[X_H] $. In the opposite direction, a map $ X_H\to X_G $ clearly induces a map $ k[X_H]\to k[X_G] $.
\end{proof}
We will develop this correspondence further in \cref{sec:Cartier_duality}.
\section{Finite constant groups}\label{sec:fin_const_gps}
Let $ \Gamma $ be a finite group, and define a functor \[ \ol\Gamma\colon \textsf{k-alg}\to \Grp \] sending every ring to $ \Gamma $, every map to the identity map. This functor is unrepresentable (not sure why), but we'll do something almost as good.

Let $ A\coloneqq Mor(\Gamma\to k) $. Let $ e_\sigma $ be the indicator function for $ \sigma\in \Gamma $, so $ \{e_\sigma\} $ is a basis for $ A $. We define a ring structure on $ A $ by $ e_\sigma^2=e_\sigma, e_\sigma e_\tau =0 $ for $ \sigma\ne\tau $, $ \sum e_\sigma =1 $.

Let $ R $ be a $ k $-algebra where the only idempotents are 0 and 1. Then a homomorphism $ \phi\colon A\to R $ is completely determined by which $ \sigma $ is sent to $ 1 $, so $ \Hom(A,R)\isoto \Gamma $ as sets. If we define a comultiplication on $ A $ by $ \Delta(e_\rho)=\sum_{\rho=\sigma\tau}e_\sigma\otimes e_\tau$, this induces the same group structure on $ \Hom(A,R) $ as that of $ \Gamma $.
If we define $ S(e_\sigma)=e_{\sigma^{-1}} $, and $ \epsilon(e_\sigma)=1 $ if $ \sigma $ is the unit of $ \Gamma $, otherwise $ 0 $, $ A $ becomes a Hopf algebra. We say that it represents the \emph{constant group scheme} $ \Gamma $.
\subsection{Scheme-theoretic aside}
As a scheme, the construction is simpler: Just take $ \Spec(\prod_{g\in \Gamma} k)=\coprod_{g\in\Gamma}\Spec k $, and define the group structure by $G\times G\ni (\Spec k)_{g,h} \mapsto (\Spec k)_{gh} \in G$. Chasing through the definitions and various categorical equivalences shows that this really is the same construction.
\section{Cartier Duality}\label{sec:Cartier_duality}
Let $ k $ be a ring, $ N $ a finite rank free $ k $-module. Recall that then: $ N\dual $ is also free; if $ M $ is also finite rank free, $(M\otimes N)\dual\isoto M\dual\otimes N\dual  $; $ \Hom_k(M\dual, N\dual)\isoto \Hom_k(M,N)\dual $; and $ \Hom, \otimes $ commute with finite direct sums.
Hence the same identities hold for finitely generated projective modules, as they are summands of free modules.

\begin{definition}\label{def:finite_group_scheme}
	Call a group scheme \emph{finite} if it is represented a $ k $-algebra $ A $ which is a finitely generated projective\footnote{The projectivity hypothesis is probably not part of the definition found elsewhere. But we will need it in these notes.} module. 
\end{definition} Note that being finitely generated as a module is a much stronger condition than being finitely generated as a $ k $-algebra.
\begin{remark}
	The scheme-theoretic statement would be: $ G $ is a group object in the category of schemes over $ k $ with a finite structure morphism.
\end{remark}
So take such a group scheme $ G $. Then we have five Hopf structure maps:
\begin{align*}
	\Delta &\colon A\to A\otimes A &\textrm{comultiplication}\\
	\epsilon&\colon A\to k &\textrm{counit}\\
	S&\colon A\to A& \textrm{coinverse}\\
	m&\colon A\times A\to A &\textrm{multiplication}\\
	u&\colon k\to A& \textrm{unit}.\\
\end{align*}
If we dualise these, we end up with something very similar:
\begin{align*}
	m\dual  &\colon A\dual\to A\dual\otimes A\dual \\
	u\dual&\colon A\dual\to k \\
	S\dual&\colon A\dual\to A\dual\\
	\Delta\dual&\colon A\dual\times A\dual\to A\dual \\
	\epsilon\dual&\colon k\to A\dual\\
\end{align*}
and we are immediately led to
\begin{theorem}[Cartier duality 1]\label{thm:Cartier_duality}
	Let $ G $ be a finite abelian group scheme represented by $ A $. Then $ A\dual $ represents another finite abelian group scheme $ G\dual $, and $ (G\dual)\dual \isoto G $.
\end{theorem}
\begin{proof}[Sketch of proof]
	This is reasonably straightforward - just check that all the axioms are satisfied, dualise diagrams if needed.
	The only tricky bit is to show that $ S\dual $ is a homomorphism: We must show that \[ \Delta\dual(S\dual\otimes S\dual) =S\dual\Delta\dual, \] which is equivalent to $ \Delta S = (S\otimes S)\Delta $.
	As group functors, this means that the following diagrams commute:

	\begin{tikzcd}
		A \arrow[r, "\Delta"] \arrow[d, "S"] & A\otimes A \arrow[d, "S\otimes S"]
		\\
		A \arrow[r, "\Delta"] & A\otimes A
	\end{tikzcd} \quad or equivalently \begin{tikzcd}
	G & G\times G \arrow[l, "mult."] \\
	G \arrow[u, "inv"'] & G\times G \arrow[l, "mult."] \arrow[u, "inv\times inv"']
\end{tikzcd} which commutes iff $ a^{-1}b^{-1}=(ab)^{-1} $ -- that is, iff $ G $ is abelian.\end{proof}

How do we compute $ G\dual(k) $?
This is the set of maps $\phi\colon A\dual\to k $, which by duality are all of the form $ \phi_b\colon f\mapsto f(b) $. We have 
\[ \phi_b(fg)=\phi_b\Delta\dual(f\otimes g)=\Delta(f\otimes g)(b)=(f\otimes g)(\Delta b) ,\] and
\[ \phi_b(f)\phi_b(g)=f(b)g(b)=(f\otimes g)(b\otimes b) \]
so $ \phi_b $ preserves products iff $ \Delta b=b\otimes b $. A similar argument for $ \epsilon $ leads us to conclude that
\[ G\dual(k)= \{\textrm{group-like elements of } A\} \]
which we know from before (\cref{thm:characters_are_gp-like_elems}) is the character group of $ G $.

We can evaluate $ G\dual(R) $ by base change. $ G_R $ is the functor represented by $ A\otimes R $, and so $ G_R\dual  $ is represented by $ (A\otimes R)\dual = A\dual\otimes R $, which also represents $ (G\dual)_R $.
Hence \[ G\dual(R)=(G\dual)_R(R)=(G_R)\dual(R)= \{\textrm{group-like elements of } A\otimes R\}.\]
We have
\begin{corollary}\label{cor:cartier_duality_2}
	Forming the dual group scheme commutes with base change, and $ G\dual(R)=\Hom(G_R, \Gm_R) $.
\end{corollary}

Note also that in this case, the group functor $ \Hom(G,H) $ is representable. This is not in general true.

\chapter{Representations of algebraic groups}
\section{Actions \& linear representations}
Let $ G $ be a group functor, $ X $ a set functor. An \emph{action} of $ G $ on $ X $ (written $ G\actson X $) is the only sensible thing: a natural transformation $ G\times X\to X $ such that all the maps $ G(R)\times X(R)\to X(R) $ are group actions.

For now, we will take $ X $ of the form $ R\mapsto V\times R $, for some $ k $-module $ R $. If the induced action of $ G(R) $ is linear, we say that we have a \emph{linear representation of $ G $ on $ V $}.
\subsubsection{Examples}
\todo{write some!} no examples yet.

Our first theorem characterizes these:
\begin{theorem}\label{thm:characterize_linear_reps}
Let $ G $ be an affine group scheme, represented by $ A $. Then linear representations of $ G $ on $ V $ correspond to $ k $-linear maps $ V\to V\otimes A $ such that
\begin{tikzcd}
	V \arrow[r, "\rho"] \arrow[d, "\rho"] & V\otimes A \arrow[d, "\id\otimes\Delta"] \\
	V\otimes A \arrow[r, "\rho\otimes \id"] & V\otimes A\otimes A
\end{tikzcd} and \begin{tikzcd}
& V \arrow[rd, two heads, hook] \arrow[ld, "\rho"] &  \\
V\otimes A \arrow[rr, "\id\otimes\epsilon"] &  & V\otimes k
\end{tikzcd} commute.
\end{theorem}
\begin{proof}
	Omitted (it's super boring). You write out all the required diagrams and find out that you need exactly these to commute.
\end{proof}

Patterned on this, we make the following
\begin{definition}\label{def:comodule}
A $ k $-module $V$, with a $ k $-linear map $ \rho\colon V\to V\otimes A $ satisfying \[(\id\otimes \epsilon)\rho =\id, \quad (\id\otimes \Delta)\rho = (\rho\otimes \id)\rho\] is an $ A $-\emph{comodule}.
\end{definition}

\begin{example}
	If $ V=A, \rho=\Delta $, this gives the \emph{regular representation} of $ G $.
\end{example}

Standard constructions on modules also work for comodules - tensor products and direct sums make sense, and we can speak of subcomodules and quotient comodules.\footnote{The category of comodules is not necessarily abelian, since kernels do not always exist. They will apparently exist if the Hopf algebra is flat. (see \href{https://mathoverflow.net/questions/94115/when-a-comodule-category-is-equivalent-to-a-module-category}{\underline{this MathOverflow discussion}.})}

\section{Finiteness}\label{sec:finiteness}
\begin{proposition}\label{prop:findim_cosubmodules}
	Let $ k $ be a field, $ A/k $ a Hopf algebra. Every $ A $-comodule $ V $ is a direct limit of finite-dimensional subcomodules.
\end{proposition}
\begin{proof}
	We show that every $ v\in V $ is contained in some finite-dimensional subcomodule. Let $ \{a_i\} $ be a basis for $ A $, so that $ \rho(v)=\sum v_i\otimes a_i $, where only finitely many $ v_i\ne0 $.
	Write $ \Delta(a_i)=\sum r_{ijk}a_j\otimes a_k $.
	Then
	\[ \sum \rho(v_i)\otimes a_i=(\rho\otimes\id)\rho(v)=(\id\otimes\Delta)(\rho(v))=\sum v_i\otimes r_{ijk}a_j\otimes a_k .\]
	We compare the coefficients of $ a_k $ to see that $ \rho(v_k)=\sum v_i\otimes r_{ijk}a_j $. Hence $ \Spn\langle v, \{v_i\}\rangle $ is a finite-dimensional subcomodule.
\end{proof}
We can extend this further to show:
\begin{theorem}\label{thm:hopfalgs_are_direct_limits}
	Any Hopf algebra $ A $ over a field $ k $ equals $ \lim_{\to}A_\alpha $, where the $ A_\alpha $ are finitely generated $ k $-Hopf-subalgebras.
\end{theorem}
\begin{proof}[Proof sketch]
	We show that every finite subset of $ A $ is in some $ A_\alpha $. From \cref{prop:findim_cosubmodules}, we know that the finite subset is in a finite-dimensional $ V $, with $ \Delta(V)\subseteq V\otimes A $. Let then $ \{v_j\} $ be a basis for $ V $. Then $ \Delta(v_j)=\sum v_i\otimes a_{ij} $.
	Then $ U\coloneqq \Spn\langle\{v_j\},\{a_{ij}\}\rangle $ satisfies $ \Delta(U)\subset U\otimes U $. Also, $ L\coloneqq \Spn\langle U, S(U)\rangle $ satisfies $ \Delta(L)\subset L\otimes L $, $ S(L)\subseteq L $. Hence $ A_\alpha = k[L] $ will work.
\end{proof}
To make full use of this, we first need a 
\begin{definition}\label{def:finite_type}
	An affine group scheme is \emph{of finite type} if its Hopf algebra is finitely generated. \footnote{Waterhouse calls this \emph{algebraic}, but \emph{finite type} is more in line with standard scheme-theoretic terminology.}
\end{definition}
Note that being \emph{of finite type} is a weaker condition than being \emph{finite}.
Then we have:
\begin{corollary}\label{cor:finite_type_is_inverse_limit}
	Every affine group scheme $ G $ over a field is an inverse limit of affine group schemes $ G_\alpha $ of finite type.
\end{corollary}

\begin{proof}
	This is immediate, simply let $ G_\alpha $ be the group schemes corresponding to the Hopf algebras $ A_\alpha $. (Or just $ \Spec A_\alpha $.)
\end{proof}

We conclude with today's final theorem:
\begin{theorem}\label{thm:everything_in_GL}
	Every affine group scheme $ G $ of finite type over a field is isomorphic to a closed subgroup of some $ \mathbf{GL}_n $.
\end{theorem}
\begin{proof}
 Let $ A $ be the Hopf algebra (the coordinate ring) of $ G $. By \cref{prop:findim_cosubmodules}, there is a finite-dimensional subcomodule $ V \subseteq A  $ containing the algebra generators. Let $ \{v_j\} $ be a basis for $ V $, such that $ \Delta v_j=\sum v_i\otimes a_{ij} $.
 
 Note that $ \Delta|V\colon V\to V\otimes A $, so $ A $ acts on $ V $. The corresponding map of Hopf algebras is 
 \[ k[\{x_{ij}\}_{i,j\le n}, {1}/{\det}]\to A, \]
 defined by $ x_{ij}\mapsto a_{ij} $.
 But $ v_j =(\epsilon\otimes\id)\Delta(v_j)=\sum\epsilon(v_i)a_{ij} $, so the image contains $ V $, thus all the algebra generators, so the image is all of $ A $. Then we have a surjective map from the Hopf algebra of $ \mathbf{GL}_{\dim V} $ onto $ A $, which means that $ G\injto \mathbf{GL}_{\dim V} $ and is closed.
\end{proof}
\chapter{Milne's book}
\section{The Identity Component of an Algebraic Group}

\begin{definition}
	The \emph{identity/neutral component} $G^\circ$ of $G$ is
	the connected component of $G$ containing $e$.
\end{definition}

We aim to show that $G^\circ$ is an algebraic subgroup of $G$.
For this, we first look at general algebraic schemes.

\begin{definition}
	And \emph{étale $k$-algebra} is a finite product of finite
	separable extensions of $k$.
\end{definition}

A few properties of étale $k$-algebras.

\begin{itemize}
	\item Finite products of étale $k$-algebras are étale.
	\item Quotients of an étale $k$-algebra are étale.
	\item Composition $A_1 \cdots A_n$ of étale $k$-subalgebras
	$A_1, \ldots, A_n$ of a $k$-algebra $A$ is an étale subalgebra of $A$.
	(Due to being a quotient of $A_1 \times \cdots \times A_n$)
\end{itemize}

\begin{definition}
	An \emph{étale $k$-scheme} $X$ is the spectrum of an étale $k$-algebra.
	They are characterized by $|X|$ being a finite set of points, and the
	local rings $\OO_{X,x}$ are finite separable field extensions of $k$.
\end{definition}

\begin{lemma}
	Let $A$ be a ring and $f\in A$ be idempotent $f^2 = f$ which is
	nontrivial (neither 0 nor 1). Then $(1-f)$ is idempotent, and $\Spec(A)$
	is the disjoint union of two open-closed subsets $D(f)$ and $D(1-f)$.
\end{lemma}

\begin{proof}
	First, $(1-f)^2= 1^2 - 2f + f^2 = 1- 2f + f = (1 - f)$, so $(1-f)$ is
	idempotent. If $f$ is nontrivial, then so is $(1-f)$.
	
	Now, note that $f(1-f) = f - f^2 = 0$. Therefore all idempotents in
	an integral domain are trivial. By considering prime ideals $P$ such
	that $A/P$ is an integral domain, we deduce that for all prime $P$,
	exactly one of $\{f, (1-f)\}$ is contained in $P$. Therefore, each
	point in $\Spec(A)$ is contained in exactly one of $D(f)$ or $D(1-f)$.
	
	Both are open, and mutual complements, so both form an open-closed
	disjoint cover of $\Spec(A)$.
\end{proof}

\begin{proposition}[1.29]
	Let $X$ be an algebraic scheme over $k$. Then there exists a largest
	étale $k$-subalgebra $\pi(X)$ in $\OO(X)$.
\end{proposition}

\begin{proof}
	Given an étale subalgebra $A$ of $\OO(X) = \Gamma(X, \OO_X)$, we know
	$A \otimes_k k^s \cong (k^s)^n$ for some $n \in \N$.
	
	In particular, we have idempotents $f_1, \ldots, f_n \in \OO(X_{k^s})$,
	(given by
	$(0, \ldots, 1, \ldots, 0)$) which are orthogonal ($f_i f_j = 0$ for
	$i \ne j$), and which add up to 1.
	
	This implies that for $i\ne j$ and for any prime $P$,
	$f_i f_j = 0 \in P$. Therefore at least one of $f_i$, $f_j$ is in $P$.
	By considering this for all $i$, $f_i \notin P$ for at most one
	$1 \le i \le n$. But $\sum f_i = 1 \notin P$, so exactly one of the
	$f_i$ is not in $P$.
	
	Therefore each point in $|X_{k^s}|$ belongs in exactly one of
	the $D(f_i)$, which form an open-closed partition of $|X_{k^s}|$.
	
	In particular, $n = [A:k] = [A \otimes k^s : k]$ is bounded above
	by the number of connected components of $|X_{k^s}|$.
	It follows that the composite of all étale subalgebras of $\OO(X)$ is
	an étale $k$-subalgebra which contains all others.
	
	
\end{proof}

Now, let $\pi_0(X) = \Spec(\pi(X))$. By the adjunction
$$\Hom_{k\text{-sch}}(X, \Spec(A)) \cong \Hom_{k\text{-alg}}(A, \OO(X))$$
the inclusion of $\pi(X)$ into
$\OO(X)$ induces a morphism $X \to \pi_0(X)$. This is universal among
morphisms of $X$ into an étale $k$-scheme (since $\pi(X)$ is the maximal
étale $k$-algebra of $\OO(X)$).

\begin{proposition}[1.30]
	Let $X$ be an algebraic scheme over $k$.
	
	For all fields $k' \supset k$,
	$$\pi_0(X_{k'}) \simeq \pi_0(X)_{k'}$$
	
	If $Y$ is a second algebraic scheme over $k$, then
	$$\pi_0(X \times Y) \simeq \pi_0(X) \times \pi_0(Y)$$
\end{proposition}

\begin{proof}
	Let $\pi = \pi(X)$ and $\pi' = \pi(X_{k'})$. Then $\pi \otimes k'$
	is an étale subalgebra of $\OO(X_{k'})$, and so by definition
	$\pi \otimes k' \subset \pi'$. The first part of the proposition
	requires us to prove we have equality.
	
	First suppose $k' = k^s$. Let $\Gamma$ be the Galois group of $k'$ over
	$k$. Then $\pi'$ is stable under $\Gamma$. From Galois Theory (A.62),
	we have $\pi'^\Gamma$ is étale over $k$ and
	$\pi'^\Gamma \otimes k' \simeq \pi'$. We also have that
	$\pi \subset \pi'^\Gamma$ and so by maximality $\pi = \pi'^\Gamma$.
	Thus $\pi \otimes k' = \pi'$.
	
	Secondly, consider $k = k^s$ and $k' = k^a$, of characteristic $p \ne 0$
	(Otherwise $k^s=k^a$). Let $e_1, \ldots, e_m$ be a basis of idempotents
	of $\pi'\cong (k')^m$ as a $k'$-vector space. Write
	$e_j = \sum a_i \otimes c_i$ for $a_i \in \OO(X)$ and $c_i \in k'$
	(since $\OO(X_{k'}) \simeq \OO(X)\otimes_k k'$). For some $r$, all
	$c_i^{p^r}$ lie in $k$, so $e_j = e_j^{p^r} =
	\sum a^{p^r} \otimes c_i^{p^r}$ (since $e_j$ is idempotent, and the
	characteristic is $p \ne 0$). This lies in $\OO(X)$ so $\OO(X)$ contains
	idempotent generators of $\pi'$, so $\pi \otimes k' = \pi'$.
	
	Thirdly, consider $k$ and $k'$ algebraically closed. Then
	$\pi\simeq k^n$ and $\pi'\simeq k'^m$; we aim to show $n=m$.
	If $m = 1$, then $\pi \otimes k' \subset \pi' = k'$, so $n=1$.
	If $n=1$, then $X$ is connected and so $X_{k'}$ is connected since
	$|X|$ is dense in $|X_{k'}|$. Thus $X$ is connected if and only if
	$X_{k'}$. Thus $n=1$ iff $m=1$. Somehow this extends?
	
	In the general case, if $\pi \otimes k' \ne \pi'$, then
	$\pi \otimes_k k' \otimes_{k'} k'^a = (\pi \otimes_k k^a)
	\otimes_{k^a} k'^a \ne \pi' \otimes k'^a$, which is a contradiction
	with the previous statements.
	
	For the other part of the proposition, we can suppose that $k = k^a$,
	and then $X \times Y$ is the union of connected subvarieties
	${x}\times Y \cup X \times {y}$ with $x \in |X|$ and $y\in |Y|$.
	
\end{proof}



\begin{proposition}[1.31]
	Let $X$ be an algebraic scheme over $k$.
	
	The fibres of the map $\phi: X \to \pi_0(X)$ are the connected
	components of $X$.
	
	For all $x \in |\pi_0(X)|$, $\phi^{-1}(x)$ is a geometrically
	connected scheme over $\kappa(x)$.
\end{proposition}

\begin{proof}
	First, if $x \in |\pi_0(X)|$, and $X_x = \phi^{-1}(x)$, then
	$\pi(X_x) = \kappa(x)$, which is a finite field extension of $k$.
	
	Now, if $\pi(X)$ is a field, then $\OO(X)$ has no non-trivial idempotents
	(as a non-trivial idempotent would give us a larger étale $k$-algebra).
	This means that $X$ is connected, as otherwise $\OO(X)$ would have a
	nontrivial idempotent.
	
	Finally, if $\pi(X)=k'$ for a finite extension $k'\supset k$, then
	$\pi(X_{k^a}) = k' \otimes_k k^a = k^a$, so $X$ is geometrically
	connected. (Not sure of this argument, we might actually have
	$\pi(X) = k$, not $k'$)
\end{proof}

\begin{corollary}[1.32]
	Let $X$ be a connected algebraic scheme over $k$, such that
	$X(k)\ne \varnothing$. Then $X$ is geometrically connected ($X_{k^a}$ is
	connected), and
	$X \times Y$ is connected for any connected algebraic scheme $Y$ over $k$.
\end{corollary}

\begin{proof}
	Let $A=\pi(X)$. By definition, $A$ is a finite product of finite
	separable extensions of $k$. If the product contains more than one term,
	then $X$ is not connected, so $A$ is a finite separable field extension
	of $k$. If $X(k)\cong \Hom(\OO(X),k)$ is non-empty, then there is
	a $k$-morphism $A \to k$ (inclusion followed by map). Therefore $A=k$,
	so $\pi(X_{k^a}) = k^a$ and so $X_{k^a}$ is connected, so $X$ is
	geometrically connected.
	
	Moreover $\pi_0(X\times Y) = \pi_0(X)\times \pi_0(Y)$, by considering
	the maximal étale $k$-algebra in
	$\OO(X\times Y) \simeq \OO(X)\otimes_k \OO(Y)$. Since $\pi(X)=k$,
	then $\pi_0(X)\times \pi_0(Y) = \pi_0(Y)$, so $X\times Y$ is connected.
\end{proof}

\begin{proposition}[1.34]
	Let $G$ be an algebraic group. The identity component of $G$ is
	an algebraic subgroup of $G$. Its formation commutes with extension
	of the base field $(G^\circ)_{k'} \simeq (G_{k'})^\circ$. In particular,
	the algebraic group $G^\circ$ is geometrically connected.
	
\end{proposition}

\begin{proof}
	The identity component $G^\circ$ of $G$ contains a $k$-point; $e$.
	Therefore it is both geometrically connected and $G^\circ\times G^\circ$
	is a connected component of $G \times G$ (1.32). As the
	multiplication maps $(e,e)\mapsto e$, then it maps $G^\circ\times G^\circ$
	to $G^\circ$. Similarly, the inversion maps $G^\circ \to G^\circ$, so
	$G^\circ$ is an algebraic subgroup of $G$.
	
	Since the formation of $G \to G^\circ$ commutes with the extension of
	the base field (1.30), then so does its fibre over $e$ (By composition of
	fibre squares). In particular,
	$(G^\circ)_{k'} = (G_{k'})^\circ$, and taking $k' = k^a$ we deduce that
	$G^\circ$ is geometrically connected.
\end{proof}

\begin{corollary}[1.35]
	Every connected component of an algebraic group is irreducible.
\end{corollary}

\begin{proof}
	Suppose for contradiction that a connected component is reducible.
	Then there must exist a point lying in at least two irreducible components
	(otherwise they'd be in different connected components).
	Thus there exists such a point in $G_{k^a}$.
	
	Now, in $G_{k^a}$, no
	irreducible component is contained in the union of the remainder
	(by definition), so there exists a point in $G_{k^a}$ which is
	contained in exactly one irreducible component.
	
	By homogeneity, all points in $G_{k^a}$ are contained in a single
	irreducible component, which contradicts the statement before.
\end{proof}

\begin{remark}
	This means that for algebraic groups, being irreducible, connected
	or geometrically connected are equivalent. This is not the case
	for schemes in general.
	
	If $G$ is affine, then the above are equivalent to the quotient
	of $\OO(G)$ by its nilradical being an integral domain.
\end{remark}

\section{The Dimension of an Algebraic Group}

\begin{definition}
	The \emph{dimension} of an algebraic group $G$ is the common dimension
	of its connected components, equal to the common Krull dimension of each
	local rings $\OO_{G,x}$ for $x \in |G|$.
\end{definition}

\begin{proof}[This is well-defined]
	The dimension $\dim(X)$ of an irreducible algebraic scheme $X$ is the
	common Krull dimension of the local rings $\OO_{X,x}$, for $x$ in $|X|$.
	Over $k^a$, such irreducible scheme $X$ becomes a finite union of
	irreducible algebraic schemes, all of the same dimension $\dim(X)$.
	
	For $G$, its irreducible components are its connected components
	(1.35). By homogeneity over $G_{k^a}$, we deduce each
	irreducible component of $G_{k^a}$ has the same dimension,
	and thus that each irreducible component of
	$G$ has the same dimension (by our previous statement).
\end{proof}

\begin{proposition}[1.37]
	For an algebraic group $G$,
	$$\dim \mathrm{Tgt}_e(G) \ge \dim G$$
	with equality if and only if $G$ is smooth.
\end{proposition}

\begin{proof}
	In general, for an algebraic $k$-scheme $G$ with a point $e$ such that
	$\kappa(e)=k$, $\dim\mathrm{Tgt}_e (G) \ge \dim G$, with equality if and only
	if $e$ is smooth on $G$ (A.52).
	
	If this is the case, then $G$ is smooth (1.28).
\end{proof}

\section{Algebraic Subgroups}

We aim to know when is $G_{\text{red}}$ an algebraic subgroup.

\begin{definition}
	A $k$-algebra $A$ is \emph{affine} if $k^a\otimes A$ is reduced
	(no non-zero nilpotent elements). If $B$ is a reduce $k$-algebra
	then $A \otimes B$ is reduced. If $k$ is perfect
	(every algebraic extension is separable), every reduced
	$k$-algebra is affine.
\end{definition}

\begin{definition}
	An algebraic scheme $X$ is \emph{geometrically reduced} if $X_{k^a}$
	is reduced. If $X$ is geometrically reduced and $Y$is reduced, then
	$X\times Y$ is reduced. If $k$ is perfect, all reduced schemes are
	geometrically reduced.
\end{definition}

\begin{proposition}[1.38-1.39]
	Let (G,m) be an algebraic group. If $G_{\text{red}}$ is geometrically
	reduced, then it is an algebraic subgroup. Alternatively, if $k$ is
	perfect, then $G_{\text{red}}$ is always an algebraic subgroup.
\end{proposition}

\begin{proof}
	Recall that if $\phi: Y \to X$ is a morphism of schemes, where $Y$
	is reduced and $Z$ is a closed subscheme of $X$, then $\phi$ factors
	through $Z_{\text{red}}$ if and only if $|\phi|$ factors through $|Z|$.
	(A.30)
	
	If $G_{\text{red}}$ is geometrically reduced, then
	$G_{\text{red}}\times G_{\text{red}}$ is reduced (by the above), so
	the multiplication restricted to  $G_{\text{red}}\times G_{\text{red}}$
	factors through $G_{\text{red}}$. Similarly the unit and inversion
	restricted to $G_{\text{red}}$ factor through $G_{\text{red}}$.
	
	It follows that $(G_{\text{red}}, m_{\text{red}})$ is an algebraic
	subgroup of $(G,m)$.
	
	If $k$ is perfect, $G_{\text{red}}$ is geometrically reduced.
\end{proof}

Note that, in general, $G_{\text{red}}$ is not an algebraic subgroup.
If it is, even if $k$ is perfect, it need not be a normal algebraic
subgroup.

An example would be $k=\mathbb{F}_p(t)$, and an algebraic subgroup of
$\mathbb{G}_a$ given by $X^{p^2}-tX^p = 0$. Then its $G_{\text{red}}$ would
be given by $X(X^{p(p-1)}-t)$, which is not geometrically reduced since
it is $X(X^{p-1}-t^{1/p})^p$ in $k^a$.

\begin{lemma}[1.40]
	Let $G$ be an algebraic group, and $S$ an abstract subgroup of $G(k)$.
	Then its Zariski closure $\bar S$ is an abstract subgroup of $G(k)$.
\end{lemma}
\begin{proof}
	Take $a,b\in \bar S$, we wish to show $ab\in S$. Take a neighborhood
	$U$ of $ab$. Since multiplication is a homeomorphism, we can take
	neighborhoods $A$ and $B$ of $a$ and $b$ such that $AB\subset U$. Then
	both $A$ and $B$ meet $S$ (since $a$, $b$ are in the closure of $S$),
	so $U$ must meet $S$. Since this happens for all neighborhoods of $ab$,
	$ab$ must lie in the Zariski closure of $S$.
	
	Since inversion is a homeomorphism, then for $a \in \bar S$, the
	inverse must lie in the closure of $S^{-1}=S$.
\end{proof}

\begin{proposition}[1.41]
	Algebraic subgroups of algebraic groups are closed
	(in the Zariski topology).
\end{proposition}

\begin{proof}
	By noting that the map $|G_{k^a}|\to |G|$ is a quotient map (surjective,
	continuous and closed), we note that $|H|$ is closed in $|G|$ if and
	only if $|H_{k^a}|$ is closed in $|G_{k^a}|$. Thus we may consider
	$k$ to be algebraically closed.
	
	Now, since the underlying topological spaces are preserved, we might
	consider both $G$ and $H$ to be reduced.
	
	Now, by definition of closure $|H|$ is open in
	$\overline{|H|}$. By lemma (1.40), $\overline{|H|}$ is also an
	abstract subgroup, so $\overline{|H|}$ is a disjoint union of cosets
	of $|H|$. Therefore $|H|$ is closed in $\overline{|H|}$, so by definition
	of closure we deduce $|H|=\overline{|H|}$, i.e. $|H|$ is closed in $|G|$.
\end{proof}

\begin{corollary}[1.42]
	The algebraic subgroups of an algebraic group satisfy the descending
	chain condition.
\end{corollary}
\begin{proof}
	This is true for closed subschemes of an algebraic scheme.
\end{proof}

\begin{corollary}[1.43]
	Algebraic subgroups of an affine algebraic group are affine.
\end{corollary}
\begin{proof}
	This is true for closed subschemes of an algebraic scheme.
\end{proof}

\begin{corollary}
	Say $H_1$, $H_2$ are algebraic subvarieties (separated, geometrically
	reduced) of an algebraic group. If $H_1(k^s)=H_2(k^s)$, then $H_1=H_2$.
\end{corollary}

\begin{proof}
	Since $H_1$ and $H_2$ are closed, we can apply (1.18):
	We say
	$X(k')$ is \emph{dense} in $X$ if the only closed subscheme such that
	$Z(k')=X(k')$ is $X$ itself. If $X$ is geometrically reduced and
	$|X(k')|$ is dense in $|X|$, then $X(k')$ is dense in $X$. This is
	because any $Z(k') = X(k')$ would have $|Z_{k'}|=|X_{k'}|$, which,
	since $X_{k'}$ is reduced by $X$ geometrically reduced, implies
	$Z_{k'}=X_{k'}$, ie $Z = X$.
	
	By considering $H_1\cap H_2 \subset H_1$ and $H_1\cap H_2 \subset H_2$,
	we deduce that since $H_1(k^s)=H_2(k^s)$ implies
	$(H_1\cap H_2)(k^s)=H_1(k^s)$, then
	$$H_1 = H_1 \cap H_2 = H_2$$
\end{proof}

\section{Normal and Characteristic Subgroups}

\section{Descent of Subgroups}
\end{document}
