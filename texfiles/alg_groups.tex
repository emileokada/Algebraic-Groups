\documentclass{memoir}

% Imports

%% Quotations (S. Gammelgaard)
\usepackage{verbatim}
\usepackage{csquotes}

%% Mathematics
\usepackage{amsfonts}
\usepackage{amsmath}
\usepackage{amssymb}    % Extra symbols
\usepackage{amsthm}     % Theorem-like environments
\usepackage{calligra}	% For the \sheafHom command
\usepackage{cancel}     % Cancel terms with \cancel, \bcancel or \xcancel
\usepackage{dsfont}     % Double stroke font with \mathds{}
\usepackage{mathtools}  % Fonts and environments for mathematical formulae
\usepackage{mathrsfs}   % Script font with \mathscr{}
\usepackage{stmaryrd}   % Brackets
\usepackage{thmtools}   % Theorem-like environments, extends amsthm

%% Graphics
\usepackage[dvipsnames,svgnames,cmyk]{xcolor}     % Pre-defined colors
\usepackage{graphicx}         % Tool for importing images
\graphicspath{{figures/}}
\usepackage{tikz}             % Drawing tool
\usetikzlibrary{calc}
\usetikzlibrary{intersections}
\usetikzlibrary{decorations.markings}
\usetikzlibrary{arrows}
\usetikzlibrary{positioning}
\usepackage{tikz-cd}		  % Commutative diagrams
\usepackage[all]{xy}

%% Organising tools
\usepackage[notref, notcite]{showkeys}               % Labels in margins
\usepackage[color= LightGray,bordercolor = LightGray,textsize    = footnotesize,figwidth    = 0.99\linewidth,obeyFinal]{todonotes} % Marginal notes

%% Misc
\usepackage{xspace}         % Clever space
\usepackage{textcomp}       % Extra symbols
\usepackage{multirow}       % Rows spanning multiple lines in tables
\usepackage{tablefootnote}  % Footnotes for tables

%% Bibliography
\usepackage[backend = biber, style = alphabetic, ibidtracker=true]{biblatex}
\addbibresource{bibliography.bib}

%% Cross references
\usepackage{varioref}

\usepackage[pdftex,hidelinks]{hyperref}
\usepackage[nameinlink, capitalize, noabbrev]{cleveref}

\pageaiv
\stockaiv

\setsecnumdepth{subsection}

\pretitle{\begin{center}\huge\sffamily\bfseries}

%% Book
\renewcommand*{\printbooktitle}[1]
{
    \hrule\vskip\onelineskip
    \centering\booktitlefont #1
    \vskip\onelineskip\hrule
}
\renewcommand*{\afterbookskip}{\par}
\renewcommand*{\booktitlefont}{\Huge\bfseries\sffamily}
\renewcommand*{\booknamefont}{\normalfont\huge\bfseries\MakeUppercase}


%% Part
\renewcommand*{\printparttitle}[1]
{
    \hrule\vskip\onelineskip
    \centering\parttitlefont #1
    \vskip\onelineskip\hrule
}
\renewcommand*{\afterpartskip}{\par}
\renewcommand*{\parttitlefont}{\Huge\bfseries\sffamily}
\renewcommand*{\partnamefont}{\normalfont\huge\bfseries\MakeUppercase}


%% Chapter 
\makeatletter
\chapterstyle{demo2}
\renewcommand*{\printchaptername}
{
    \centering\chapnamefont\MakeUppercase{\@chapapp}
}
\renewcommand*{\printchapternum}{\chapnumfont\thechapter\space}
\renewcommand*{\chaptitlefont}{\Huge\bfseries\sffamily\center}
\let\ps@chapter\ps@empty


%% Lower level sections
\setsecheadstyle{\Large\bfseries\sffamily\raggedright}
\setsubsecheadstyle{\large\bfseries\sffamily\raggedright}
\setsubsubsecheadstyle{\normalsize\bfseries\sffamily\raggedright}
\setparaheadstyle{\normalsize\bfseries\sffamily\raggedright}
\setsubparaheadstyle{\normalsize\bfseries\sffamily\raggedright}


%% Abstract
\renewcommand{\abstractnamefont}{\sffamily\bfseries}


%% Header
\pagestyle{ruled}
\makeevenhead{ruled}{\sffamily\leftmark}{}{}
\makeoddhead {ruled}{}{}{\sffamily\rightmark}


%% Trim marks
\trimLmarks

%% Environments
\declaretheorem[style = plain, numberwithin = section]{theorem}
\declaretheorem[style = plain,      sibling = theorem]{corollary}
\declaretheorem[style = plain,      sibling = theorem]{lemma}
\declaretheorem[style = plain,      sibling = theorem]{proposition}
\declaretheorem[style = plain,      sibling = theorem]{observation}
\declaretheorem[style = plain,      sibling = theorem]{conjecture}
\declaretheorem[style = definition, sibling = theorem]{definition}
\declaretheorem[style = definition, sibling = theorem]{example}
\declaretheorem[style = definition, sibling = theorem]{notation}
\declaretheorem[style = remark,     sibling = theorem]{remark}
\crefname{observation}{Observation}{Observations}
\Crefname{observation}{Observation}{Observations}
\crefname{conjecture}{Conjecture}{Conjectures}
\Crefname{conjecture}{Conjecture}{Conjectures}
\crefname{notation}{Notation}{Notations}
\Crefname{notation}{Notation}{Notations}
\crefname{diagram}{Diagram}{Diagrams}
\Crefname{diagram}{Diagram}{Diagrams}

%% Operators
\DeclareMathOperator{\Spn}{Span}				% Span of vectors
\DeclareMathOperator{\Gal}{Gal}					% Galois group
\DeclareMathOperator{\Spec}{Spec}				% Spectrum
\DeclareMathOperator{\Proj}{Proj}				% Proj construction
\DeclareMathOperator{\Gr}{\mathbb{G}}			% Grassmannian
\DeclareMathOperator{\Aut}{Aut}					% Automorphisms
\DeclareMathOperator{\End}{End}					% Endomorphisms
\DeclareMathOperator{\CH}{CH}					% Chow ring/group
\DeclareMathOperator{\CHr}{CH^\bullet}			% Chow ring
\DeclareMathOperator{\Cox}{Cox}					% Cox ring
\DeclareMathOperator{\Div}{Div}					% Divisor group
\DeclareMathOperator{\Cl}{Cl}					% Class group
\DeclareMathOperator{\Pic}{Pic}					% Picard group
\DeclareMathOperator{\relSpec}{\mathbf{Spec}}	% Relative Spec
\DeclareMathOperator{\relProj}{\mathbf{Proj}}	% Relative Proj
\DeclareMathOperator{\ord}{ord}					% Order
\DeclareMathOperator{\res}{res}					% Residue
\DeclareMathOperator{\coker}{coker}				% Cokernel (\ker is already defined)
\DeclareMathOperator{\im}{im}					% Image
\DeclareMathOperator{\coim}{coim}			    % Coimage
\DeclareMathOperator{\tr}{tr}					% Trace
\DeclareMathOperator{\rk}{rk}					% Rank
\DeclareMathOperator{\Hom}{Hom}					% Homomorphisms
\DeclareMathOperator{\cl}{cl}					% Class map
\DeclareMathOperator{\sheafHom}					% Sheaf of homomorphisms
{
    \mathscr{H}\text{\kern -5.2pt {\calligra\large om}}\,
}
\DeclareMathOperator{\codim}{codim}				% Codimension
\DeclareMathOperator{\Sym}{Sym}					% Symmetric powers
\DeclareMathOperator{\II}{I\!I}					% Second fundamental form
\DeclareMathOperator{\Pfaff}{Pfaff}				% Pfaffian

%% Delimiters
\DeclarePairedDelimiter{\p}{\lparen}{\rparen}          % Parenthesis
\DeclarePairedDelimiter{\set}{\lbrace}{\rbrace}        % Set
\DeclarePairedDelimiter{\abs}{\lvert}{\rvert}          % Absolute value
\DeclarePairedDelimiter{\norm}{\lVert}{\rVert}         % Norm
\DeclarePairedDelimiter{\ip}{\langle}{\rangle}         % Inner product, ideal
\DeclarePairedDelimiter{\sqb}{\lbrack}{\rbrack}        % Square brackets
\DeclarePairedDelimiter{\ssqb}{\llbracket}{\rrbracket} % Double brackets
\DeclarePairedDelimiter{\ceil}{\lceil}{\rceil}         % Ceiling
\DeclarePairedDelimiter{\floor}{\lfloor}{\rfloor}      % Floor
\DeclarePairedDelimiter{\tuple}{\langle}{\rangle}		% Tuple	


%% Sets
\newcommand{\N}{\mathbb{N}}    						% Natural numbers
\newcommand{\Z}{\mathbb{Z}}    						% Integers
\newcommand{\Q}{\mathbb{Q}}    						% Rational numbers
\newcommand{\R}{\mathbb{R}}    						% Real numbers
\newcommand{\C}{\mathbb{C}}    						% Complex numbers
\newcommand{\A}{\mathbb{A}}    						% Affine space
\renewcommand{\P}{\mathbb{P}}  						% Projective space
%Additions (S. Gammelgaard)
\renewcommand{\H}{\mathbb{H}}						% Hyperbolic space, or half-plane
\newcommand{\D}{\mathbb{D}} 						% Unit disk
\newcommand{\F}{\mathbb{F}} 						% Field
\newcommand{\bP}[1]{\mathbf{P}\!\left(#1\right)}	% Projectivisation of bundles

%% Special groups and Lie groups
\newcommand{\GL}{\mathbf{GL}}						% General linear group
\newcommand{\PGL}{\mathbf{PGL}}						% Projective linear group
\newcommand{\PSL}{\mathbf{PSL}}						% Projective linear group
\newcommand{\SL}{\mathbf{SL}}						% Special linear group
\newcommand{\SO}{\mathbf{SO}}						% Special linear group
\newcommand{\Mat}{\mathbf{Mat}}						% Special linear group

%% Lie algebras
\newcommand{\lalg}[1]{{\normalfont\mathfrak{#1}}}	% General for Lie algebras
\newcommand{\gl}{\lalg{gl}}							% General linear algebra
%\newcommand{\sl}{\lalg{sl}}							% Special linear algebra

%% Cones of cycles on varieties and related objects
\newcommand{\NS}{\mathrm{NS}}						% Neron-Severi group
\newcommand{\Nef}{\mathrm{Nef}}						% Nef cone
\newcommand{\NE}{\mathrm{NE}}						% Cone of curves
\newcommand{\Eff}{\mathrm{Eff}}						% Effective cone
\newcommand{\Pseff}{\mathrm{PSeff}}					% Pseudoeffective cone

%% Categories
\newcommand{\cat}[1]{{\normalfont\mathsf{#1}}}	% General for categories
\newcommand{\Cat}{\cat{Cat}}					% Category of categories
\newcommand{\Sch}{\cat{Sch}}					% Schemes
\newcommand{\Set}{\cat{Set}}					% Sets
\newcommand{\Grp}{\cat{Grp}}					% Groups
\newcommand{\AbGrp}{\cat{AbGrp}}				% AbGroups
\newcommand{\Ab}{\cat{Ab}}      				% AbGroups
\newcommand{\Ring}{\cat{Ring}}					% Rings
\newcommand{\Top}{\cat{Top}}					% Topological spaces
\newcommand{\Alg}{\cat{Alg}}					% Algebras
\newcommand{\SMan}{\cat{Man}^\infty}			% Smooth manifolds
\newcommand{\Coh}[1]{\cat{Coh}({#1})}			% Coherent sheaves
\newcommand{\QCoh}[1]{\cat{QCoh}({#1})}			% Quasi-coherent sheaves
\newcommand{\Fun}{\cat{Fun}}					% Category of functors
\newcommand{\PreSh}{\cat{PreSh}}			    % Category of presheaves
\newcommand{\Sh}{\cat{Sh}}			            % Category of presheaves

%% Miscellaneous mathematics
\newcommand{\ol}[1]{\overline{#1}}							% Overline
\newcommand{\Dirsum}{\bigoplus}								% Direct sum
\newcommand{\shf}[1]{\mathscr{#1}}							% Sheaf
\newcommand{\OO}{\mathcal{O}}								% Structure sheaf
\DeclareMathOperator{\id}{id}								% Identity
\newcommand{\tens}[1]{\otimes_{#1}}							% Tensor product
\newcommand{\normal}{\vartriangleleft}						% Normal subgroup, ideal of ring or Lie algebra
\newcommand{\lamron}{\vartriangleright}						% The opposite of above
\newcommand{\dvol}{d\operatorname{vol}}						% Volume form on a KÀhler manifold
\newcommand{\cha}{\operatorname{char}}						% Characteristic of a field
\newcommand{\Hilb}{\operatorname{Hilb}}						% Hilbert scheme
\newcommand{\isoto}{\xrightarrow{\sim}}						% Isomorphism
\newcommand{\injto}{\xhookrightarrow{}}						% Injective map
\newcommand{\ratto}{\dashrightarrow}						% Rational map
\newcommand{\rateq}{\overset{\sim}{\ratto}}					% Rational equivalence
\newcommand{\Bl}[2]{\operatorname{Bl}_{#2} #1}				% Blow-up of #1 along #2
%\newcommand{\Bl}[2]{#1\kern -2pt \uparrow_{#2}}			% 	(alternativ som ingen andre liker, buhu)
\newcommand{\fracpart}[2]{\frac{\partial #1}{\partial #2}}	% Partial derivative
\renewcommand{\setminus}{\smallsetminus}
\newcommand{\transp}[1]{{}^t#1}								% transposed map, Voisin-style
\newcommand{\dual}{{}^\vee}									% dual of map, vector bundle, sheaf, etc...
\newcommand{\littletilde}{\tilde}							% for the next
\renewcommand{\tilde}{\widetilde}
\newcommand{\Spe}{\text{Sp\'e}}						        % Etale space
\newcommand{\Gm}{{\mathbb{G}_m}}							% Multiplicative group
\newcommand{\Ga}{\mathbb{G}_a}								% Additive group
\newcommand{\actson}{\curvearrowright}						% Group acting on

%%\newcommand{\dual}{{}^{\smash{\scalebox{.7}[1.4]{\rotatebox{90}{\guilsinglleft}}}}}	% Dual of sheaf/vector space et cetera

%% Miscellaneous, not-strictly-mathematical
\renewcommand{\qedsymbol}{\(\blacksquare\)}
\newcommand{\ie}{\leavevmode\unskip, i.e.,\xspace}
\newcommand{\eg}{\leavevmode\unskip, e.g.,\xspace}
%\newcommand{\wlog}{\leavevmode\unskip without loss of generality \xspace}
\newcommand{\dash}{\textthreequartersemdash\xspace}
\newcommand{\TikZ}{Ti\textit{k}Z\xspace}
\newcommand{\matlab}{\textsc{Matlab}\xspace}


\title{Algebraic groups}
\author{Emile T. Okada}

\begin{document}
\maketitle
\tableofcontents
\chapter{Affine group schemes}
\section{Affine schemes}
\begin{definition}
    Let $k$ be a commutative ring. An \textit{affine scheme over $k$} is a representable functor $F:\Alg_k\to \Set$.
\end{definition}
Let us try to motivate this definition a bit. 
Let $k$ be a commutative ring and $\{f_\alpha\}\subseteq k[x_1,\dots,x_n]$.
For any $k$-algebra $R$ we can consider the set of points in $R^n$ satisfying $f_\alpha = 0$ for all $\alpha$.
Let us call this set $I(R)$.
It is clear that if we have a $k$-algebra homomorphism $f:R\to S$ we obtain a map $I(f):I(R) \to I(S)$, and that this turns $I$ into a functor $I:\Alg_k\to \Set$.
\begin{thm}
    Let $J$ be the ideal of $k[x_1,\dots,x_n]$ generated by the $f_\alpha$ and $A = k[x_1,\dots,x_n]/J$.
    Then $I$ is a representable functor with representative $A$.
\end{thm}
\begin{proof}
    Let $a = (a_1,\dots,a_n)\in I(R)$ and let $f_a:k[x_1,\dots,x_n]\to R$ be the $k$-algebra homomorphism given by $x_i\mapsto a_i$.
    By definition of the point $a$, the ideal $J$ must map to 0 under $f_a$.
    Thus $f_a$ factors through $A$ and we obtain a map $\bar f_a: A\to R$.

    Conversely, given a map $f:A\to R$, let $a = (f(\bar x_1),\dots,f(\bar x_n))\in R^n$ where $\bar x_i$ is the image of $x_i$ in $A$.
    Since $f$ is a homomorphism it is clear that $a$ lies in $I(R)$.

    These two maps give a bijection between the sets $I(R)$ and $\Hom_k(A,R)$.
    It is straightforward to check this is a natural bijection and so $I$ is naturally isomorphic to the functor $\Hom(A,-)$.
\end{proof}
This theorem says that the affine schemes are exactly what we expect them to be.
\subsection{The Yoneda lemma}
Recall from category theory the following result on representable functors.
\begin{thm}
    (Yoneda's lemma).
    Let $F:\mathcal C\to \Set$ be a functor, $\mathcal C$ be locally small and $c\in \mathcal C$. Then there is a natrual bijection
    \begin{align}
        \Hom(\mathcal C(c,-),F) &\leftrightarrow Fc \\
        \eta \quad &\rightarrow \quad \eta_c(\id_c) \nonumber \\
        \eta_x: f\mapsto F(f)(y) \quad &\leftarrow \quad y. \nonumber 
    \end{align}
\end{thm}
\begin{corollary}
    (Yoneda embedding).
    The functor 
    \begin{equation}
        \mathcal C(\bullet,-):\mathcal C^{op}\to \Fun(\mathcal C, \Set)
    \end{equation}
    is full and faithful.
\end{corollary}
\begin{proof}
    Let $c,d\in \mathcal C$.
    From the Yoneda lemma we have a bijection
    \begin{equation}
        \Hom(\mathcal C(d,-),\mathcal C(c,-)) \leftrightarrow \mathcal C(c,d).
    \end{equation}
    Let $f\in \mathcal C(c,d)$. 
    Under the bijection this gets sent to the natural transformation $\eta_x: g\mapsto \mathcal C(c,-)(g)(f) = g\circ f$.
    Thus the backwards map in the Yoneda bijection is just $\mathcal C(\bullet, -)$ and so the result follows.
\end{proof}
\begin{remark}
    It follows that $\mathcal C^{op}$ is equivalent to the category of representable functors from $\mathcal C$.
    In fact there exists a functor $P:\Fun^{rep}(\mathcal C,\Set) \to \mathcal C^{op}$ such that $P\circ \mathcal C(\bullet,-) = \id_{\mathcal C^{op}}$, $\mathcal C(\bullet,-)\circ P \cong \id_{\Fun^{rep}(\mathcal C,\Set)}$ and $\mathcal C(\bullet,-)\circ P|_{\im(\mathcal C(\bullet,-)} = \id_{\im(\mathcal C(\bullet,-)}$.
\end{remark}
\begin{corollary}
    Let $c,d\in \mathcal C$. 
    Then $c\cong d$ if and only if $\mathcal C(c,-) \cong \mathcal C(d,-)$.
\end{corollary}
\begin{proof}
    $(\Rightarrow)$ This follows by functoriality.

    $(\Leftarrow)$ Let $\alpha: \mathcal C(c,-) \Rightarrow \mathcal C(d,-)$ be an isomorphism and $\beta$ its inverse.
    Let $a:d\to c$ and $b:c\to d$ be the corresponding maps.
    By naturality of the Yoneda bijection
    \begin{equation}
        \begin{tikzcd}
            \Hom(\mathcal C(c,-),\mathcal C(d,-)) \arrow[r] \arrow[d,"\beta\circ"] & \mathcal C(d,c) \arrow[d,"\beta_c"] \\
            \Hom(\mathcal C(c,-),\mathcal C(c,-)) \arrow[r] & \mathcal C(c,c) 
        \end{tikzcd}
    \end{equation}
    commutes.
    So $\beta_c(\alpha_c(\id_c)) = \id_c$.
    But $\beta_c(\alpha_c(\id_c)) = a\circ b$ and so $a\circ b = \id_c$.
    Similarly $b\circ a = \id_b$ and so the result follows.

    Alternatively note use the remark. 
\end{proof}
This last corollary implies that two $k$-algebras are isomorphic if and only if the corresponding affine schemes are.
\section{Affine group schemes}
\begin{definition}
    An \textit{affine group scheme over $k$} is a functor $F:\Alg_k\to \Grp$ such that $F$ composed with the forgetful functor $\Grp \to \Set$ is representable.
\end{definition}
\begin{example}
    Let $k = \Z$ (so $\Alg_k = \Ring$).
    Then $\SL_n:\Ring\to \Grp$, $R\mapsto \SL_n(R)$ is an affine group scheme over $\Z$.
\end{example}
\begin{proposition}
    Let $E,F,G$ be affine group schemes over $k$ represented by $A,B,C$ respectively.
    If we have morphisms $E\to G$ and $F\to G$ then the pullback exists and is represented by $A\otimes_CB$.
\end{proposition}
\begin{proof}
    Existence follows from the fact that we can compute the pullback pointwise and $\Grp$ has pullbacks.
    Explicitly 
    \begin{equation}
        (E\times_G F)(R) = \{(e,f) : \text{ $e$ and $f$ have same image in $G(R)$}\}.
    \end{equation}
    For the second part of the proposition note that the pullback in $\Alg_k^{op}$ is of the required form and that we have an equivalence of categories between $\Alg_k^{op}$ and affine schemes over $k$.
\end{proof}
\begin{definition}
    Let $F$ be an affine group scheme over $k$ and let $\phi:k\to k'$ be a ring homomorphism.
    Any $k'$-algebra can be turned into a $k$ by composing by $\phi$ and so we can turn $F$ into a functor on $k'$-algebras.
    Call this new functor $F_{k'}$.
\end{definition}
\begin{proposition}
    If $F$ is represented by $A$ then $F_{k'}$ then is represented by $A\otimes_kk'$.
\end{proposition}
\begin{proof}
    There is a natural bijection 
    \begin{equation}
        \Hom_{k'}(A\otimes_kk',S) \leftrightarrow \Hom_k(A,S).
    \end{equation}
\end{proof}
\subsection{Hopf algebras}
Another way to define an affine group scheme is that it is a representable functor $F:\Alg_k \to \Set$ together with natural transformations $\mu:F\times F\to F$, $i:F\to F$ and $u:e\to G$ where $e$ is the functor $\Hom_k(k,-)$ such that
\begin{equation}
    \begin{tikzcd}
        F\times F\times F \arrow[r,"\id\times\mu"] \arrow[d,"\mu\times\id"] & F\times F \arrow[d,"\mu"] \\
        F\times F \arrow[r,"\mu"] & F
    \end{tikzcd}
\end{equation}
\begin{equation}
    \begin{tikzcd}
        e\times F \arrow[r,"u\times\id"] \arrow[dr,"\cong"] & F\times F \arrow[d,"\mu"] & F\times e \arrow[r,"\id\times u"] \arrow[dr,"\cong"] & F\times F \arrow[d,"\mu"] \\
                                                            & F & & F
    \end{tikzcd}
\end{equation}
\begin{equation}
    \begin{tikzcd}
        F \arrow[r,shift right,swap,"i\times\id"] \arrow[r,shift left,"\id\times i"] \arrow[d] & F\times F \arrow[d,"\mu"] \\
        e \arrow[r,"u"] & F
    \end{tikzcd}
\end{equation}
all commute.
If $F$ is represented by $A$ then translating back to $\Alg_k$ we obtain maps $\Delta:A\to A\otimes_k A$, $S:A\to A$ and $\epsilon:A\to k$.
Satisfying the same commutative diagrams, but with arrows reversed and $\mu$ replaced with $\Delta$ etc.
The algebra $A$ together with these maps is what we call a Hopf algebra.
\begin{definition}
    Let $\psi:H\to G$ be a morphism of affine groups schemes. 
    We say $\psi$ is a \textit{closed embedding} if the corresponding map on algebras is surjective.
    $H$ is then isomorphic to a closed subgroup of $G$ represented by the corresponding quotient of the algebra of $A$.
\end{definition}
\begin{thm}
    Affine group schemes over $k$ correspond to Hopf algebras over $k$.
\end{thm}
\begin{definition}
    Let $A$ be a $k$-algebra. 
    We call an ideal $I$ of $A$ a \textit{Hopf ideal} if $A/I$ inherits the structure of a Hopf algebra.
\end{definition}
\begin{proposition}
    Let $A$ be a $k$-algebra and $I\normal A$. Then $I$ is a hopf ideal if and only if $\Delta(I)\subseteq I\otimes A + A\otimes I$, $S(I)\subseteq I$ and $\epsilon(I) = 0$.
\end{proposition}
\begin{proof}
    Todo.
\end{proof}
\begin{definition}
    Let $\Phi:G\to H$ be a morphism of affine groups schemes. 
    Then $\ker\Phi (R) = \ker(G(R)\to H(R))$ or alternatively $\ker\Phi = G\times_H\{e\}$.
    It follows that if $G,H$ are represented by $A,B$ respectively then $\ker\Phi$ is represented by $A\otimes_B k\cong A/AI_B$.
\end{definition}
\subsection{Characters}
\begin{definition}
    A homomorphism $G\to G_m$ is called a character.
\end{definition}
\begin{thm}
    The characters of an affine group scheme $G$ represented by $A$ correspond to the group like elements of $A$.
\end{thm}
\end{document}
