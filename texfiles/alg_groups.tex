\documentclass{memoir}

\input{header}
\title{Algebraic groups}
\author{Emile T. Okada}

\begin{document}
\maketitle
\tableofcontents
\chapter{Affine group schemes}
\section{Affine schemes}
\begin{definition}
    Let $k$ be a commutative ring. An \textit{affine scheme over $k$} is a representable functor $F:\Alg_k\to \Set$.
\end{definition}
Let us try to motivate this definition a bit. 
Let $k$ be a commutative ring and $\{f_\alpha\}\subseteq k[x_1,\dots,x_n]$.
For any $k$-algebra $R$ we can consider the set of points in $R^n$ satisfying $f_\alpha = 0$ for all $\alpha$.
Let us call this set $I(R)$.
It is clear that if we have a $k$-algebra homomorphism $f:R\to S$ we obtain a map $I(f):I(R) \to I(S)$, and that this turns $I$ into a functor $I:\Alg_k\to \Set$.
\begin{thm}
    Let $J$ be the ideal of $k[x_1,\dots,x_n]$ generated by the $f_\alpha$ and $A = k[x_1,\dots,x_n]/J$.
    Then $I$ is a representable functor with representative $A$.
\end{thm}
\begin{proof}
    Let $a = (a_1,\dots,a_n)\in I(R)$ and let $f_a:k[x_1,\dots,x_n]\to R$ be the $k$-algebra homomorphism given by $x_i\mapsto a_i$.
    By definition of the point $a$, the ideal $J$ must map to 0 under $f_a$.
    Thus $f_a$ factors through $A$ and we obtain a map $\bar f_a: A\to R$.

    Conversely, given a map $f:A\to R$, let $a = (f(\bar x_1),\dots,f(\bar x_n))\in R^n$ where $\bar x_i$ is the image of $x_i$ in $A$.
    Since $f$ is a homomorphism it is clear that $a$ lies in $I(R)$.

    These two maps give a bijection between the sets $I(R)$ and $\Hom_k(A,R)$.
    It is straightforward to check this is a natural bijection and so $I$ is naturally isomorphic to the functor $\Hom(A,-)$.
\end{proof}
This theorem says that the affine schemes are exactly what we expect them to be.
\subsection{The Yoneda lemma}
Recall from category theory the following result on representable functors.
\begin{thm}
    (Yoneda's lemma).
    Let $F:\mathcal C\to \Set$ be a functor, $\mathcal C$ be locally small and $c\in \mathcal C$. Then there is a natrual bijection
    \begin{align}
        \Hom(\mathcal C(c,-),F) &\leftrightarrow Fc \\
        \eta \quad &\rightarrow \quad \eta_c(\id_c) \nonumber \\
        \eta_x: f\mapsto F(f)(y) \quad &\leftarrow \quad y. \nonumber 
    \end{align}
\end{thm}
\begin{corollary}
    (Yoneda embedding).
    The functor 
    \begin{equation}
        \mathcal C(\bullet,-):\mathcal C^{op}\to \Fun(\mathcal C, \Set)
    \end{equation}
    is full and faithful.
\end{corollary}
\begin{proof}
    Let $c,d\in \mathcal C$.
    From the Yoneda lemma we have a bijection
    \begin{equation}
        \Hom(\mathcal C(d,-),\mathcal C(c,-)) \leftrightarrow \mathcal C(c,d).
    \end{equation}
    Let $f\in \mathcal C(c,d)$. 
    Under the bijection this gets sent to the natural transformation $\eta_x: g\mapsto \mathcal C(c,-)(g)(f) = g\circ f$.
    Thus the backwards map in the Yoneda bijection is just $\mathcal C(\bullet, -)$ and so the result follows.
\end{proof}
\begin{remark}
    It follows that $\mathcal C^{op}$ is equivalent to the category of representable functors from $\mathcal C$.
    In fact there exists a functor $P:\Fun^{rep}(\mathcal C,\Set) \to \mathcal C^{op}$ such that $P\circ \mathcal C(\bullet,-) = \id_{\mathcal C^{op}}$, $\mathcal C(\bullet,-)\circ P \cong \id_{\Fun^{rep}(\mathcal C,\Set)}$ and $\mathcal C(\bullet,-)\circ P|_{\im(\mathcal C(\bullet,-)} = \id_{\im(\mathcal C(\bullet,-)}$.
\end{remark}
\begin{corollary}
    Let $c,d\in \mathcal C$. 
    Then $c\cong d$ if and only if $\mathcal C(c,-) \cong \mathcal C(d,-)$.
\end{corollary}
\begin{proof}
    $(\Rightarrow)$ This follows by functoriality.

    $(\Leftarrow)$ Let $\alpha: \mathcal C(c,-) \Rightarrow \mathcal C(d,-)$ be an isomorphism and $\beta$ its inverse.
    Let $a:d\to c$ and $b:c\to d$ be the corresponding maps.
    By naturality of the Yoneda bijection
    \begin{equation}
        \begin{tikzcd}
            \Hom(\mathcal C(c,-),\mathcal C(d,-)) \arrow[r] \arrow[d,"\beta\circ"] & \mathcal C(d,c) \arrow[d,"\beta_c"] \\
            \Hom(\mathcal C(c,-),\mathcal C(c,-)) \arrow[r] & \mathcal C(c,c) 
        \end{tikzcd}
    \end{equation}
    commutes.
    So $\beta_c(\alpha_c(\id_c)) = \id_c$.
    But $\beta_c(\alpha_c(\id_c)) = a\circ b$ and so $a\circ b = \id_c$.
    Similarly $b\circ a = \id_b$ and so the result follows.

    Alternatively note use the remark. 
\end{proof}
This last corollary implies that two $k$-algebras are isomorphic if and only if the corresponding affine schemes are.
\section{Affine group schemes}
\begin{definition}
    An \textit{affine group scheme over $k$} is a functor $F:\Alg_k\to \Grp$ such that $F$ composed with the forgetful functor $\Grp \to \Set$ is representable.
\end{definition}
\begin{example}
    Let $k = \Z$ (so $\Alg_k = \Ring$).
    Then $\SL_n:\Ring\to \Grp$, $R\mapsto \SL_n(R)$ is an affine group scheme over $\Z$.
\end{example}
\begin{proposition}
    Let $E,F,G$ be affine group schemes over $k$ represented by $A,B,C$ respectively.
    If we have morphisms $E\to G$ and $F\to G$ then the pullback exists and is represented by $A\otimes_CB$.
\end{proposition}
\begin{proof}
    Existence follows from the fact that we can compute the pullback pointwise and $\Grp$ has pullbacks.
    Explicitly 
    \begin{equation}
        (E\times_G F)(R) = \{(e,f) : \text{ $e$ and $f$ have same image in $G(R)$}\}.
    \end{equation}
    For the second part of the proposition note that the pullback in $\Alg_k^{op}$ is of the required form and that we have an equivalence of categories between $\Alg_k^{op}$ and affine schemes over $k$.
\end{proof}
\begin{definition}
    Let $F$ be an affine group scheme over $k$ and let $\phi:k\to k'$ be a ring homomorphism.
    Any $k'$-algebra can be turned into a $k$ by composing by $\phi$ and so we can turn $F$ into a functor on $k'$-algebras.
    Call this new functor $F_{k'}$.
\end{definition}
\begin{proposition}
    If $F$ is represented by $A$ then $F_{k'}$ then is represented by $A\otimes_kk'$.
\end{proposition}
\begin{proof}
    There is a natural bijection 
    \begin{equation}
        \Hom_{k'}(A\otimes_kk',S) \leftrightarrow \Hom_k(A,S).
    \end{equation}
\end{proof}
\subsection{Hopf algebras}
Another way to define an affine group scheme is that it is a representable functor $F:\Alg_k \to \Set$ together with natural transformations $\mu:F\times F\to F$, $i:F\to F$ and $u:e\to G$ where $e$ is the functor $\Hom_k(k,-)$ such that
\begin{equation}
    \begin{tikzcd}
        F\times F\times F \arrow[r,"\id\times\mu"] \arrow[d,"\mu\times\id"] & F\times F \arrow[d,"\mu"] \\
        F\times F \arrow[r,"\mu"] & F
    \end{tikzcd}
\end{equation}
\begin{equation}
    \begin{tikzcd}
        e\times F \arrow[r,"u\times\id"] \arrow[dr,"\cong"] & F\times F \arrow[d,"\mu"] & F\times e \arrow[r,"\id\times u"] \arrow[dr,"\cong"] & F\times F \arrow[d,"\mu"] \\
                                                            & F & & F
    \end{tikzcd}
\end{equation}
\begin{equation}
    \begin{tikzcd}
        F \arrow[r,shift right,swap,"i\times\id"] \arrow[r,shift left,"\id\times i"] \arrow[d] & F\times F \arrow[d,"\mu"] \\
        e \arrow[r,"u"] & F
    \end{tikzcd}
\end{equation}
all commute.
If $F$ is represented by $A$ then translating back to $\Alg_k$ we obtain maps $\Delta:A\to A\otimes_k A$, $S:A\to A$ and $\epsilon:A\to k$.
Satisfying the same commutative diagrams, but with arrows reversed and $\mu$ replaced with $\Delta$ etc.
The algebra $A$ together with these maps is what we call a Hopf algebra.
\begin{definition}
    Let $\psi:H\to G$ be a morphism of affine groups schemes. 
    We say $\psi$ is a \textit{closed embedding} if the corresponding map on algebras is surjective.
    $H$ is then isomorphic to a closed subgroup of $G$ represented by the corresponding quotient of the algebra of $A$.
\end{definition}
\begin{thm}
    Affine group schemes over $k$ correspond to Hopf algebras over $k$.
\end{thm}
\begin{definition}
    Let $A$ be a $k$-algebra. 
    We call an ideal $I$ of $A$ a \textit{Hopf ideal} if $A/I$ inherits the structure of a Hopf algebra.
\end{definition}
\begin{proposition}
    Let $A$ be a $k$-algebra and $I\normal A$. Then $I$ is a hopf ideal if and only if $\Delta(I)\subseteq I\otimes A + A\otimes I$, $S(I)\subseteq I$ and $\epsilon(I) = 0$.
\end{proposition}
\begin{proof}
    Todo.
\end{proof}
\begin{definition}
    Let $\Phi:G\to H$ be a morphism of affine groups schemes. 
    Then $\ker\Phi (R) = \ker(G(R)\to H(R))$ or alternatively $\ker\Phi = G\times_H\{e\}$.
    It follows that if $G,H$ are represented by $A,B$ respectively then $\ker\Phi$ is represented by $A\otimes_B k\cong A/AI_B$.
\end{definition}
\subsection{Characters}
\begin{definition}
    A homomorphism $G\to G_m$ is called a character.
\end{definition}
\begin{thm}
    The characters of an affine group scheme $G$ represented by $A$ correspond to the group like elements of $A$.
\end{thm}
\end{document}
